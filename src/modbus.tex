\documentclass[russian,utf8,pointsection]{eskdtext}
\usepackage{cmap}			% Поиск по русским словам в конечном pdf документе
\usepackage{eskdchngsheet}
\usepackage[T2A]{fontenc}
\usepackage{pscyr}			% Подключение "красивых" шрифтов киррилицы
\usepackage{amstext}
\usepackage{amsmath}
\usepackage{listings}
\usepackage{color}
\usepackage{ifthen}

% работа с сылками
\usepackage{hyperref}
% \usepackage[usenames,dvipsnames,svgnames,table,rgb]{xcolor}
\hypersetup{			
	unicode=true,           				% русские буквы в раздела PDF
	pdftitle={Заголовок},   				% Заголовок
	pdfauthor={Автор},      				% Автор
	pdfsubject={Тема},      				% Тема
	pdfcreator={Создатель}, 				% Создатель
	pdfproducer={Производитель}, 			% Производитель
	pdfkeywords={keyword1} {key2} {key3}, 	% Ключевые слова
	colorlinks=true,      					% false: ссылки в рамках; true: цветные ссылки
	linkcolor=black,        				% внутренние ссылки
	citecolor=black,        				% на библиографию
	filecolor=black,        				% на файлы
	urlcolor=black          				% на URL
}

% Дает доступ к командам:
% \MakeTextUppercase{} - сделать все символы заглавными
\usepackage{textcase} 

%Изменение отображения Содержания
%\makeatletter
%\renewcommand{\l@section}{\@dottedtocline{1}{0em}{1.25em}}
%\renewcommand{\l@subsection}{\@dottedtocline{2}{1.25em}{1.75em}}
%\renewcommand{\l@subsubsection}{\@dottedtocline{3}{2.75em}{2.6em}}
%\makeatother

% Работа с таблицами
% p{} - top align, m{} - middle align, b{} - bottom align
\usepackage{ltablex} 										% longtable с функциональностю tabularx
\usepackage{multirow} 										% Слияние строк в таблице
\renewcommand{\tabularxcolumn}[1]{>{\arraybackslash}m{#1}}	% выравнивание в ячейке таблицы по середине по вертикали
\newcolumntype{M}[1]{>{\centering \arraybackslash}m{#1}} 	% колонка с заданной шириной и выравниванием по центру
\newcolumntype{Z}{>{\centering \arraybackslash} X} 			% колонка с выравниванием по центру

% текущее устройство
\def \deviceCurrent	{АВАНТ РЗСК}

% возможные устройства
\def \deviceK 		{АВАНТ К400}
\def \deviceR		{АВАНТ Р400}
\def \deviceRM		{АВАНТ Р400м}
\def \deviceRZSK	{АВАНТ РЗСК}


\newif\ifCommand % проверка наличия Команд
\newif\ifDefense % првоерка наличия Защиты

% формирование переменных для разных устройств 
\ifx \deviceCurrent \deviceK  
	\def \deviceDecimal {ПБКМ.424325.005 РЭ3} 
	\def \deviceTitle	{ПРИЕМОПЕРЕДАТЧИК КОМАНД РЕЛЕЙНОЙ ЗАЩИТЫ И ПРОТИВОАВАРИЙНОЙ	АВТОМАТИКИ АВАНТ К400} 
	\def \deviceOKP		{ОКП 66 5730}
	\Commandtrue
	\Defensefalse
	% Неисправности общие
	\def \deviceGlobalErrorI 	{Неисправность чтения FLASH}	% 0x0001
	\def \deviceGlobalErrorII 	{Неисправность записи FLASH}	% 0x0002
	\def \deviceGlobalErrorIII 	{Неисправность чтения PLIS}		% 0x0004
	\def \deviceGlobalErrorIV	{Неисправность записи PLIS}		% 0x0008
	\def \deviceGlobalErrorV	{Неисправность записи 2RAM}		% 0x0010
	\def \deviceGlobalErrorVI	{Резерв}						% 0x0020
	\def \deviceGlobalErrorVII	{Резерв}						% 0x0040
	\def \deviceGlobalErrorVIII	{Резерв}						% 0x0080
	\def \deviceGlobalErrorIX	{Неисправность DSP 2RAM}		% 0x0100
	\def \deviceGlobalErrorX	{Неисправность чтения 2RAM}		% 0x0200
	\def \deviceGlobalErrorXI	{Резерв}						% 0x0400
	\def \deviceGlobalErrorXII	{Низкое напряжение выхода}		% 0x0800
	\def \deviceGlobalErrorXIII	{Высокое напряжение выхода}		% 0x1000
	\def \deviceGlobalErrorXIV	{Резерв}						% 0x2000
	\def \deviceGlobalErrorXV	{Резерв}						% 0x4000
	\def \deviceGlobalErrorXVI	{Резерв}						% 0x8000
	% Предупреждения общие
	\def \deviceGlobalWarnI 	{Установите часы}				% 0x0001
	\def \deviceGlobalWarnII 	{Резерв}						% 0x0002
	\def \deviceGlobalWarnIII 	{Резерв}						% 0x0004
	\def \deviceGlobalWarnIV	{Резерв}						% 0x0008
	\def \deviceGlobalWarnV		{Резерв}						% 0x0010
	\def \deviceGlobalWarnVI	{Резерв}						% 0x0020
	\def \deviceGlobalWarnVII	{Резерв}						% 0x0040
	\def \deviceGlobalWarnVIII	{Резерв}						% 0x0080
	\def \deviceGlobalWarnIX	{Резерв}						% 0x0100
	\def \deviceGlobalWarnX		{Резерв}						% 0x0200
	\def \deviceGlobalWarnXI	{Резерв}						% 0x0400
	\def \deviceGlobalWarnXII	{Резерв}						% 0x0800
	\def \deviceGlobalWarnXIII	{Резерв}						% 0x1000
	\def \deviceGlobalWarnXIV	{Резерв}						% 0x2000
	\def \deviceGlobalWarnXV	{Резерв}						% 0x4000
	\def \deviceGlobalWarnXVI	{Резерв}						% 0x8000
	% Неисправности приемника
	\def \devicePrmErrorI 		{Нет блока БСК}					% 0x0001
	\def \devicePrmErrorII 		{Неисправность версии БСК}		% 0x0002
	\def \devicePrmErrorIII 	{Нет КЧ}						% 0x0004
	\def \devicePrmErrorIV		{Резерв}						% 0x0008
	\def \devicePrmErrorV		{Резерв}						% 0x0010
	\def \devicePrmErrorVI		{Резерв}						% 0x0020
	\def \devicePrmErrorVII		{Резерв}						% 0x0040
	\def \devicePrmErrorVIII	{Резерв}						% 0x0080
	\def \devicePrmErrorIX		{Неисправность записи БСК}		% 0x0100
	\def \devicePrmErrorX		{Неисправность выключения КСК}	% 0x0200
	\def \devicePrmErrorXI		{Неисправность включения КСК}	% 0x0400
	\def \devicePrmErrorXII		{Неисправность контроля команд}	% 0x0800
	\def \devicePrmErrorXIII	{Резерв}						% 0x1000
	\def \devicePrmErrorXIV		{Резерв}						% 0x2000
	\def \devicePrmErrorXV		{Резерв}						% 0x4000
	\def \devicePrmErrorXVI		{Резерв}						% 0x8000
	% Предупреждения приемника
	\def \devicePrmWarnI 		{Снижение уровня ПРМ}			% 0x0001
	\def \devicePrmWarnII 		{Ошибка работы ЦС}				% 0x0002
	\def \devicePrmWarnIII 		{Вход RX ЦС пуст}				% 0x0004
	\def \devicePrmWarnIV		{Резерв}						% 0x0008
	\def \devicePrmWarnV		{Резерв}						% 0x0010
	\def \devicePrmWarnVI		{Резерв}						% 0x0020
	\def \devicePrmWarnVII		{Резерв}						% 0x0040
	\def \devicePrmWarnVIII		{Резерв}						% 0x0080
	\def \devicePrmWarnIX		{Резерв}						% 0x0100
	\def \devicePrmWarnX		{Резерв}						% 0x0200
	\def \devicePrmWarnXI		{Резерв}						% 0x0400
	\def \devicePrmWarnXII		{Резерв}						% 0x0800
	\def \devicePrmWarnXIII		{Резерв}						% 0x1000
	\def \devicePrmWarnXIV		{Резерв}						% 0x2000
	\def \devicePrmWarnXV		{Резерв}						% 0x4000
	\def \devicePrmWarnXVI		{Резерв}						% 0x8000 
	% Неисправности передатчика
	\def \devicePrdErrorI 		{Нет блока БСК}					% 0x0001
	\def \devicePrdErrorII 		{Неисправность версии БСК}		% 0x0002
	\def \devicePrdErrorIII 	{Резерв}						% 0x0004
	\def \devicePrdErrorIV		{Резерв}						% 0x0008
	\def \devicePrdErrorV		{Резерв}						% 0x0010
	\def \devicePrdErrorVI		{Резерв}						% 0x0020
	\def \devicePrdErrorVII		{Резерв}						% 0x0040
	\def \devicePrdErrorVIII	{Резерв}						% 0x0080
	\def \devicePrdErrorIX		{Неисправность чтения команд}	% 0x0100
	\def \devicePrdErrorX		{Неисправность выключения Теста}% 0x0200
	\def \devicePrdErrorXI		{Неисправность включения Теста}	% 0x0400
	\def \devicePrdErrorXII		{Неисправность входов команд}	% 0x0800
	\def \devicePrdErrorXIII	{Резерв}						% 0x1000
	\def \devicePrdErrorXIV		{Резерв}						% 0x2000
	\def \devicePrdErrorXV		{Резерв}						% 0x4000
	\def \devicePrdErrorXVI		{Резерв}						% 0x8000
	% Предупреждения передатчика
	\def \devicePrdWarnI 		{Резерв}						% 0x0001
	\def \devicePrdWarnII 		{Ошибка работы ЦС}				% 0x0002
	\def \devicePrdWarnIII 		{Вход RX ЦС пуст}				% 0x0004
	\def \devicePrdWarnIV		{Резерв}						% 0x0008
	\def \devicePrdWarnV		{Резерв}						% 0x0010
	\def \devicePrdWarnVI		{Резерв}						% 0x0020
	\def \devicePrdWarnVII		{Резерв}						% 0x0040
	\def \devicePrdWarnVIII		{Резерв}						% 0x0080
	\def \devicePrdWarnIX		{Резерв}						% 0x0100
	\def \devicePrdWarnX		{Резерв}						% 0x0200
	\def \devicePrdWarnXI		{Резерв}						% 0x0400
	\def \devicePrdWarnXII		{Резерв}						% 0x0800
	\def \devicePrdWarnXIII		{Резерв}						% 0x1000
	\def \devicePrdWarnXIV		{Резерв}						% 0x2000
	\def \devicePrdWarnXV		{Резерв}						% 0x4000
	\def \devicePrdWarnXVI		{Резерв}						% 0x8000 
\else \ifx \deviceCurrent \deviceR  
	\def \deviceDecimal {ПБКМ.424325.001 РЭ3} 
	\def \deviceTitle	{ПРИЕМОПЕРЕДАТЧИК СИГНАЛОВ РЕЛЕЙНОЙ ЗАЩИТЫ АВАНТ Р400}
	\def \deviceOKP		{ОКП 42 3211}
	\Commandfalse
	\Defensetrue
\else \ifx \deviceCurrent \deviceRM 
	\def \deviceDecimal {ПБКМ.424325.001 РЭ3}
	\def \deviceTitle	{ПРИЕМОПЕРЕДАТЧИК СИГНАЛОВ РЕЛЕЙНОЙ ЗАЩИТЫ АВАНТ Р400М}
	\def \deviceOKP		{ОКП 42 3211}
	\Commandfalse
	\Defensetrue
	% Неисправности общие
	\def \deviceGlobalErrorI 	{Неисправность чтения FLASH}	% 0x0001
	\def \deviceGlobalErrorII 	{Неисправность записи FLASH}	% 0x0002
	\def \deviceGlobalErrorIII 	{Неисправность чтения PLIS}		% 0x0004
	\def \deviceGlobalErrorIV	{Неисправность записи PLIS}		% 0x0008
	\def \deviceGlobalErrorV	{Неисправность записи 2RAM}		% 0x0010
	\def \deviceGlobalErrorVI	{АК-нет ответа}					% 0x0020
	\def \deviceGlobalErrorVII	{АК-снижение запаса}			% 0x0040
	\def \deviceGlobalErrorVIII	{Помеха в линии}				% 0x0080
	\def \deviceGlobalErrorIX	{Неисправность DSP 2RAM}		% 0x0100
	\def \deviceGlobalErrorX	{Неисправность чтения 2RAM}		% 0x0200
	\def \deviceGlobalErrorXI	{Ток покоя}						% 0x0400
	\def \deviceGlobalErrorXII	{Низкое напряжение выхода}		% 0x0800
	\def \deviceGlobalErrorXIII	{Высокое напряжение выхода}		% 0x1000
	\def \deviceGlobalErrorXIV	{Неисправность МкУМ}			% 0x2000
	\def \deviceGlobalErrorXV	{ВЧ тракт восстановлен}			% 0x4000
	\def \deviceGlobalErrorXVI	{Резерв}						% 0x8000
	% Предупреждения общие
	\def \deviceGlobalWarnI 	{Установите часы}				% 0x0001
	\def \deviceGlobalWarnII 	{Резерв}						% 0x0002
	\def \deviceGlobalWarnIII 	{Резерв}						% 0x0004
	\def \deviceGlobalWarnIV	{Резерв}						% 0x0008
	\def \deviceGlobalWarnV		{Резерв}						% 0x0010
	\def \deviceGlobalWarnVI	{Резерв}						% 0x0020
	\def \deviceGlobalWarnVII	{Резерв}						% 0x0040
	\def \deviceGlobalWarnVIII	{Резерв}						% 0x0080
	\def \deviceGlobalWarnIX	{Резерв}						% 0x0100
	\def \deviceGlobalWarnX		{Резерв}						% 0x0200
	\def \deviceGlobalWarnXI	{Резерв}						% 0x0400
	\def \deviceGlobalWarnXII	{Резерв}						% 0x0800
	\def \deviceGlobalWarnXIII	{Резерв}						% 0x1000
	\def \deviceGlobalWarnXIV	{Резерв}						% 0x2000
	\def \deviceGlobalWarnXV	{Резерв}						% 0x4000
	\def \deviceGlobalWarnXVI	{Резерв}						% 0x8000
	% Неисправности защиты
	\def \deviceDefenseErrorI 	{Нет блока БСЗ}					% 0x0001
	\def \deviceDefenseErrorII 	{Неисправность версии БСЗ}		% 0x0002
	\def \deviceDefenseErrorIII {Неисправность переключателей}	% 0x0004
	\def \deviceDefenseErrorIV	{Резерв}						% 0x0008
	\def \deviceDefenseErrorV	{АК-нет ответа}					% 0x0010
	\def \deviceDefenseErrorVI	{Резерв}						% 0x0020
	\def \deviceDefenseErrorVII	{Неисправность удаленного ДФЗ}	% 0x0040
	\def \deviceDefenseErrorVIII{Неисправность удаленного ВЫХ}	% 0x0080
	\def \deviceDefenseErrorIX	{Неисправность входа ПУСК}		% 0x0100
	\def \deviceDefenseErrorX	{Неисправность входа СТОП}		% 0x0200
	\def \deviceDefenseErrorXI	{Удаленный без ответа}			% 0x0400
	\def \deviceDefenseErrorXII	{Неисправность цепи ВЫХ}		% 0x0800
	\def \deviceDefenseErrorXIII{Удаленный обнаружил помеху}	% 0x1000
	\def \deviceDefenseErrorXIV	{Неисправность записи ВЫХ}		% 0x2000
	\def \deviceDefenseErrorXV	{Длительная помеха}				% 0x4000
	\def \deviceDefenseErrorXVI	{Неисправность ДФЗ}				% 0x8000
	% Предупреждения защиты
	\def \deviceDefenseWarnI 	{АК-снижение запаса}			% 0x0001
	\def \deviceDefenseWarnII 	{Нет сигнала МАН}				% 0x0002
	\def \deviceDefenseWarnIII 	{Порог по помехе}				% 0x0004
	\def \deviceDefenseWarnIV	{Автоконтроль}					% 0x0008
	\def \deviceDefenseWarnV	{Резерв}						% 0x0010
	\def \deviceDefenseWarnVI	{Резерв}						% 0x0020
	\def \deviceDefenseWarnVII	{Резерв}						% 0x0040
	\def \deviceDefenseWarnVIII	{Резерв}						% 0x0080
	\def \deviceDefenseWarnIX	{Резерв}						% 0x0100
	\def \deviceDefenseWarnX	{Резерв}						% 0x0200
	\def \deviceDefenseWarnXI	{Резерв}						% 0x0400
	\def \deviceDefenseWarnXII	{Резерв}						% 0x0800
	\def \deviceDefenseWarnXIII	{Резерв}						% 0x1000
	\def \deviceDefenseWarnXIV	{Резерв}						% 0x2000
	\def \deviceDefenseWarnXV	{Резерв}						% 0x4000
	\def \deviceDefenseWarnXVI	{Резерв}						% 0x8000 
\else \ifx \deviceCurrent \deviceRZSK 
	\def \deviceDecimal {ПБКМ.424325.004 РЭ3}
	\def \deviceTitle	{ПРИЕМОПЕРЕДАТЧИК СИГНАЛОВ И КОМАНД РЕЛЕЙНОЙ ЗАЩИТЫ АВАНТ РЗСК} 
	\def \deviceOKP		{ОКП 42 3211}
	\Commandtrue
	\Defensetrue
	% Неисправности общие
	\def \deviceGlobalErrorI 	{Неисправность чтения FLASH}	% 0x0001
	\def \deviceGlobalErrorII 	{Неисправность записи FLASH}	% 0x0002
	\def \deviceGlobalErrorIII 	{Неисправность чтения PLIS}		% 0x0004
	\def \deviceGlobalErrorIV	{Неисправность записи PLIS}		% 0x0008
	\def \deviceGlobalErrorV	{Неисправность записи 2RAM}		% 0x0010
	\def \deviceGlobalErrorVI	{Резерв}						% 0x0020
	\def \deviceGlobalErrorVII	{Резерв}						% 0x0040
	\def \deviceGlobalErrorVIII	{Резерв}						% 0x0080
	\def \deviceGlobalErrorIX	{Неисправность DSP 2RAM}		% 0x0100
	\def \deviceGlobalErrorX	{Неисправность чтения 2RAM}		% 0x0200
	\def \deviceGlobalErrorXI	{Резерв}						% 0x0400
	\def \deviceGlobalErrorXII	{Низкое напряжение выхода}		% 0x0800
	\def \deviceGlobalErrorXIII	{Высокое напряжение выхода}		% 0x1000
	\def \deviceGlobalErrorXIV	{Резерв}						% 0x2000
	\def \deviceGlobalErrorXV	{Резерв}						% 0x4000
	\def \deviceGlobalErrorXVI	{Резерв}						% 0x8000
	% Предупреждения общие
	\def \deviceGlobalWarnI 	{Резерв}						% 0x0001
	\def \deviceGlobalWarnII 	{Резерв}						% 0x0002
	\def \deviceGlobalWarnIII 	{Резерв}						% 0x0004
	\def \deviceGlobalWarnIV	{Резерв}						% 0x0008
	\def \deviceGlobalWarnV		{Резерв}						% 0x0010
	\def \deviceGlobalWarnVI	{Резерв}						% 0x0020
	\def \deviceGlobalWarnVII	{Резерв}						% 0x0040
	\def \deviceGlobalWarnVIII	{Резерв}						% 0x0080
	\def \deviceGlobalWarnIX	{Резерв}						% 0x0100
	\def \deviceGlobalWarnX		{Резерв}						% 0x0200
	\def \deviceGlobalWarnXI	{Установите часы}				% 0x0400
	\def \deviceGlobalWarnXII	{Резерв}						% 0x0800
	\def \deviceGlobalWarnXIII	{Резерв}						% 0x1000
	\def \deviceGlobalWarnXIV	{Резерв}						% 0x2000
	\def \deviceGlobalWarnXV	{Резерв}						% 0x4000
	\def \deviceGlobalWarnXVI	{Резерв}						% 0x8000
	% Неисправности защиты
	\def \deviceDefenseErrorI 	{Нет блока БСЗ}					% 0x0001
	\def \deviceDefenseErrorII 	{Неисправность версии БСЗ}		% 0x0002
	\def \deviceDefenseErrorIII {Неисправность переключателей}	% 0x0004
	\def \deviceDefenseErrorIV	{Резерв}						% 0x0008
	\def \deviceDefenseErrorV	{Резерв}						% 0x0010
	\def \deviceDefenseErrorVI	{Резерв}						% 0x0020
	\def \deviceDefenseErrorVII	{Резерв}						% 0x0040
	\def \deviceDefenseErrorVIII{Резерв}						% 0x0080
	\def \deviceDefenseErrorIX	{Неисправность входа ПУСК}		% 0x0100
	\def \deviceDefenseErrorX	{Неисправность входа СТОП}		% 0x0200
	\def \deviceDefenseErrorXI	{Резерв}						% 0x0400
	\def \deviceDefenseErrorXII	{Неисправность цепи ВЫХ}		% 0x0800
	\def \deviceDefenseErrorXIII{Резерв}						% 0x1000
	\def \deviceDefenseErrorXIV	{Неисправность записи ВЫХ}		% 0x2000
	\def \deviceDefenseErrorXV	{Нет сигнала РЗ}				% 0x4000
	\def \deviceDefenseErrorXVI	{Удаленный неисправен}			% 0x8000
	% Предупреждения защиты
	\def \deviceDefenseWarnI 	{Низкий уровень РЗ}				% 0x0001
	\def \deviceDefenseWarnII 	{Нет сигнала МАН}				% 0x0002
	\def \deviceDefenseWarnIII 	{Резерв}						% 0x0004
	\def \deviceDefenseWarnIV	{Резерв}						% 0x0008
	\def \deviceDefenseWarnV	{Резерв}						% 0x0010
	\def \deviceDefenseWarnVI	{Резерв}						% 0x0020
	\def \deviceDefenseWarnVII	{Резерв}						% 0x0040
	\def \deviceDefenseWarnVIII	{Резерв}						% 0x0080
	\def \deviceDefenseWarnIX	{Резерв}						% 0x0100
	\def \deviceDefenseWarnX	{Резерв}						% 0x0200
	\def \deviceDefenseWarnXI	{Резерв}						% 0x0400
	\def \deviceDefenseWarnXII	{Резерв}						% 0x0800
	\def \deviceDefenseWarnXIII	{Резерв}						% 0x1000
	\def \deviceDefenseWarnXIV	{Резерв}						% 0x2000
	\def \deviceDefenseWarnXV	{Резерв}						% 0x4000
	\def \deviceDefenseWarnXVI	{Резерв}						% 0x8000 
	% Неисправности приемника
	\def \devicePrmErrorI 		{Нет блока БСК}					% 0x0001
	\def \devicePrmErrorII 		{Неисправность версии БСК}		% 0x0002
	\def \devicePrmErrorIII 	{Нет КС}						% 0x0004
	\def \devicePrmErrorIV		{Резерв}						% 0x0008
	\def \devicePrmErrorV		{Резерв}						% 0x0010
	\def \devicePrmErrorVI		{Резерв}						% 0x0020
	\def \devicePrmErrorVII		{Резерв}						% 0x0040
	\def \devicePrmErrorVIII	{Резерв}						% 0x0080
	\def \devicePrmErrorIX		{Неисправность записи БСК}		% 0x0100
	\def \devicePrmErrorX		{Неисправность выключения команд}% 0x0200
	\def \devicePrmErrorXI		{Неисправность включения команд}% 0x0400
	\def \devicePrmErrorXII		{Неисправность контроля команд}	% 0x0800
	\def \devicePrmErrorXIII	{Резерв}						% 0x1000
	\def \devicePrmErrorXIV		{Резерв}						% 0x2000
	\def \devicePrmErrorXV		{Резерв}						% 0x4000
	\def \devicePrmErrorXVI		{Резерв}						% 0x8000
	% Предупреждения приемника
	\def \devicePrmWarnI 		{Снижение уровня ПРМ}			% 0x0001
	\def \devicePrmWarnII 		{Резерв}						% 0x0002
	\def \devicePrmWarnIII 		{Резерв}						% 0x0004
	\def \devicePrmWarnIV		{Резерв}						% 0x0008
	\def \devicePrmWarnV		{Резерв}						% 0x0010
	\def \devicePrmWarnVI		{Резерв}						% 0x0020
	\def \devicePrmWarnVII		{Резерв}						% 0x0040
	\def \devicePrmWarnVIII		{Резерв}						% 0x0080
	\def \devicePrmWarnIX		{Резерв}						% 0x0100
	\def \devicePrmWarnX		{Резерв}						% 0x0200
	\def \devicePrmWarnXI		{Резерв}						% 0x0400
	\def \devicePrmWarnXII		{Резерв}						% 0x0800
	\def \devicePrmWarnXIII		{Резерв}						% 0x1000
	\def \devicePrmWarnXIV		{Резерв}						% 0x2000
	\def \devicePrmWarnXV		{Резерв}						% 0x4000
	\def \devicePrmWarnXVI		{Резерв}						% 0x8000 
	% Неисправности передатчика
	\def \devicePrdErrorI 		{Нет блока БСК}					% 0x0001
	\def \devicePrdErrorII 		{Неисправность версии БСК}		% 0x0002
	\def \devicePrdErrorIII 	{Резерв}						% 0x0004
	\def \devicePrdErrorIV		{Резерв}						% 0x0008
	\def \devicePrdErrorV		{Резерв}						% 0x0010
	\def \devicePrdErrorVI		{Резерв}						% 0x0020
	\def \devicePrdErrorVII		{Резерв}						% 0x0040
	\def \devicePrdErrorVIII	{Резерв}						% 0x0080
	\def \devicePrdErrorIX		{Неисправность чтения команд}	% 0x0100
	\def \devicePrdErrorX		{Неисправность выключения Теста}% 0x0200
	\def \devicePrdErrorXI		{Неисправность включения Теста}	% 0x0400
	\def \devicePrdErrorXII		{Неисправность входов команд}	% 0x0800
	\def \devicePrdErrorXIII	{Резерв}						% 0x1000
	\def \devicePrdErrorXIV		{Резерв}						% 0x2000
	\def \devicePrdErrorXV		{Резерв}						% 0x4000
	\def \devicePrdErrorXVI		{Резерв}						% 0x8000
	% Предупреждения передатчика
	\def \devicePrdWarnI 		{Резерв}						% 0x0001
	\def \devicePrdWarnII 		{Ошибка работы ЦС}				% 0x0002
	\def \devicePrdWarnIII 		{Вход RX ЦС пуст}				% 0x0004
	\def \devicePrdWarnIV		{Резерв}						% 0x0008
	\def \devicePrdWarnV		{Резерв}						% 0x0010
	\def \devicePrdWarnVI		{Резерв}						% 0x0020
	\def \devicePrdWarnVII		{Резерв}						% 0x0040
	\def \devicePrdWarnVIII		{Резерв}						% 0x0080
	\def \devicePrdWarnIX		{Резерв}						% 0x0100
	\def \devicePrdWarnX		{Резерв}						% 0x0200
	\def \devicePrdWarnXI		{Резерв}						% 0x0400
	\def \devicePrdWarnXII		{Резерв}						% 0x0800
	\def \devicePrdWarnXIII		{Резерв}						% 0x1000
	\def \devicePrdWarnXIV		{Резерв}						% 0x2000
	\def \devicePrdWarnXV		{Резерв}						% 0x4000
	\def \devicePrdWarnXVI		{Резерв}						% 0x8000 	
\else 
	\def \deviceDecimal	{\textcolor{red}{ОШИБКА ДЕЦИМАЛЬНОГО НОМЕРА!}}
	\def \deviceTitle	{\textcolor{red}{ОШИБКА ТИТУЛА!}}
	\def \deviceOKP		{\textcolor{red}{ОШИБКА ОКП!}}
\fi \fi \fi \fi

% Заполнение граф Титульного листа и основной надписи
\ESKDcompany{ООО <<Прософт-Системы>>}
\ESKDclassCode{\deviceOKP}
\ESKDtitle{\deviceTitle}
\ESKDdocName{ Руководство по использованию протокола MODBUS}
\ESKDsignature{\deviceDecimal{}}
\ESKDgroup{\normalsize ООО <<Прософт-Системы>>}

% основная надпись
\ESKDauthor{\ESKDfontII Щеблыкин М.В.}
\ESKDchecker{\ESKDfontII Макаров Е.Г.}
\ESKDnormContr{\ESKDfontII Бунина О.Ю.}
\ESKDapprovedBy{\ESKDfontII Чирков А.Г.}
\ESKDdate{2014/08/14}
\ESKDcolumnI{АВАНТ \deviceCurrent{} \\ \vspace{0.5cm} \ESKDfontIII{Инструкция 
	по использованию протокола MODBUS}}

% Добавлено отображение Города на Титульном листе
\renewcommand{\ESKDtheTitleFieldX}{Екатеринбург \\ \ESKDtheYear}

% Уменьшен размер шрифта для заголовков секций
\ESKDsectStyle{section}{\large \bfseries \MakeTextUppercase}

% Выравнивание по центру.
% Предназначено для выравнивания надписей в шапке таблицы
\newcommand{\calign}[1]{\centering #1 \arraybackslash} 

\begin{document}
	\maketitle
	\tableofcontents
	
	\newpage
	
	%%% ----------
\section{Общие сведения} \label{sec:overview}

Приемопередатчик команд релейной защиты и~противоаварийной автоматики \deviceCurrent{} может быть подключен к~информационно-управляющей сети с~помощью коммуникационного прота RS-485/RS-422 и~поддерживаемого протокола MODBUS. 

Реализованный в~приемопередатчике протокол соотвествует стандартному протоколу MODBUS.

В~сети MODBUS приемопередатчик \deviceCurrent{} всегда является ведомым
(slave) устройством.

Реализованный протокол MODBUS обеспечивает доступ к~большинству переменных и~флагов приемопередатчика. Список и~описание доступных параметров приведен в~разделе~\ref{sec:map}.

Функции протокола реализованы в~блоке БСП (плата БСП-ПИ). Выводы подключения находятся на~клеммнике КВП.
	%%% ----------
\section{Технические характеристики} \label{sec:tth}

\begin{list}{--}{}
\item Физический уровень: последовательный канал стандарта RS-485/RS-422.
\item Максимальная длина линии <<управляющее устройство - <<\deviceCurrent{}>> определеятеся типом кабеля и~скоростью передачи. Рекомендуется использовать экранированную витую пару.
\item Скорости обмена данными: 600, 1200, 2400, 4800, 9600, 19200 бит/с.
\item Адреса устройства в сети MODBUS: от 1 до 247.
\item Канальный уровень: RTU режим.
\item Прикладной уровень: поддерживаются следующие стандартные функции MODBUS: 01~(01h), 03~(03h), 05~(05h), 06~(06h), 15~(0Fh), 16~(10h), 17~(11h).
\end{list}
	%%% ----------
\section{Установка параметров соединения} \label{sec:setup}

Параметры соединения должны быть настроены до~установки связи. 

Параметры физической линии:
\begin{list}{--}{}
	\item \textit{Baud Rate} "--- скорость передачи данных, бит/с; 
	\item \textit{Parity} "--- способ использования бита четности (<<нет>>, <<чет>>, <<нечет>>); 
	\item \textit{CL} "--- длина поля данных в посылке последовательного порта (7 или 8~бит); 
	\item \textit{SBL} --- длина поля стоп-битов в~посылке последовательного порта (1 или 2~бита);
\end{list}

Параметры протокола:
\begin{list}{--}{}
	\item \textit{Адрес в~локальной сети} - уникальный идентификационный номер устройства в~сети MODBUS.
\end{list}

Параметр физической линии \textit{CL} в~приемопередатчике равен 8~бит, и не~может быть изменен. При включенной проверке бита четности \textit{Parity} (<<чет>> или <<нечет>>), количество стоп-битов \textit{SBL} должно  быть равно~1. Если бит четности \textit{Parity} не~используется (<<нет>>), количество стоп-битов \textit{SBL} должно быть~2.  

Настройки со~стороны ведущего (master) устройства должны совпадать с~настройками приемопередатчика.

Параметр \textit{Адрес в локальной сети} приемопередатчика должен иметь уникальное, отличное от~любого другого устройства в~сети MODBUS значение в~диапазоне от~1 до~247.

Настройка параметров соединения производится с~пульта управления блока БСП в~меню <<Настройка/Интерфейс>>. Параметры соединения хранятся в~ПЗУ и не~требуют повторной настройки при~следующем включении питания.

Изменение параметров соединения с~пульта управления при~установленном соединении приводит к~потере связи.

В~таблице \ref{tab:connection} приведены возможные значения параметров и значения, установленные на~предприятии-изготовителе.

\begin{tabularx}{\linewidth}{| M{2.5cm} | X | M{4cm} | M{2cm} |}
	\caption{Параметры соединения} \label{tab:connection} \\
    
    \hline
    Название параметра	& \calign{Описание параметра}	& Возможные значения & Значение \\ \hline 
    \endfirsthead
    
    \multicolumn{4}{r}{продолжение следует\ldots} \\ 
    \endfoot 
	\endlastfoot
	
	\hline
	Название параметра		& \calign{Описание параметра}												& Возможные значения 													& Значение 	\\ \hline 
	\endhead
	Адрес в локальной сети 	& Задает уникальный идентификационный номер приемопередатчика в~сети MODBUS & от~1 до~247 															& 1			\\ \hline
	Режим MODBUS			& Позволяет выбрать формат кадра протокола MODBUS 							& только RTU															& RTU		\\ \hline
	Baud Rate				& Определяет скорость передачи данных в~сети MODBUS 						& 600, 1200, 2400, 4800, 9600, 19200 бит/с								& 19200		\\ \hline
	Parity 					& Определяет способ использования бита четности в~байтах кадра				& чет (дополнение до~четности), нечет (дополнение до~нечетности), нет 	& чет		\\ \hline
	CL						& Длина поля данных в посылке последовательного порта 						& не~настраиваются 														& 8~бит		\\ \hline
	SBL						& Длина поля стоп-битов в посылке последовательного порта 					& 1 или 2	 															& 1~бит		\\ \hline
\end{tabularx}


	%%%----------
\section{Функции и коды исключения Modbus} \label{sec:func}

В~таблице \ref{tab:function} приведены поддерживаемые программным обеспечением приемопередатчика  стандартные функции MODBUS. Подробное описание этих функций можно найти в~документе «MODBUS Application Protocol Specification» и по~адресу \url{www.modbus.org}.

\begin{tabularx}{\linewidth}{|X|M{2cm}|}
	\caption{Поддерживаемые стандартные функции MODBUS}  \label{tab:function}	\\ 
    
    \hline
    \centering Описание функции \arraybackslash	& Код функции	\\ \hline
    \endfirsthead
    
    \hline
    \multicolumn{2}{r}{продолжение следует\ldots} \\
    \endfoot
	\endlastfoot
	
	\multicolumn{2}{l}{Продолжение таблицы \ref{tab:function}} \\ \hline
	\centering Описание функции \arraybackslash																		& Код функции \\
	\endhead    
	
    Чтение внутренних битовых флагов приемопередатчика.			    												& 01 (01h)  \\ \hline
    Чтение 16-битовых регистров внешних входных сигналов или внутренних 16-битовых регистров устройства. 			& 03 (03h)	\\ \hline
    Запись битового значения во~внутренний битовый флаг приемопередатчика.											& 05 (05h)	\\ \hline	
    Запись значения в~один 16-битовый регистр внешних сигналов или 16-битовый внутренний регистр устройства.		& 06 (06h)	\\ \hline
    Запись значений битовых значений во~внутренние битовые флаги приемопередатчика.									& 10 (0Ah)	\\ \hline
    Запись значений в~16-битовые регистры внешних входных сигналов или внутренние 16-битовые регистры устройства.   & 16 (10h)	\\ \hline
    Чтение идентификатора подчиненного устройства.   																& 17 (11h)	\\ \hline
\end{tabularx}

Количество регистров в~одном запросе не~должно превышать~32, а флагов "---~256.

В таблице \ref{tab:code_error} приведены коды исключения - сообщения об~ошибках, возвращаемые приемопередатчиком в~ответ на~некорректный запрос со~стороны ведущего устройства.

\begin{tabularx}{\linewidth}{| M{1.2cm} | M{4.5cm} | X |}
	\caption{Коды исключения} \label{tab:code_error}	\\ 	 
	
    \hline
    Код	& \centering Имя \arraybackslash	& \calign{Описание кода исключения} \\ \hline
    \endfirsthead
    
    \multicolumn{3}{r}{продолжение следует\ldots} \\
    \endfoot
	\endlastfoot
	
	\multicolumn{3}{l}{Продолжение таблицы \ref{tab:code_error}} 	\\ \hline
	Код 		& Имя									& \calign{Описание кода исключения} \\ \hline
	\endhead	
	01 (01h)	& ILLEGAL FUNCTION						& Код функции, принятой в~запросе, не~поддерживается ведомым устройством. Это означает, что запрашиваемая функция не~поддерживается приемопередатчиком.	\\ \hline
	02 (02h)	& ILLEGAL DATA ADDRESS					& Адрес данных, принятый в~запросе, недоступен в~ведомом устройстве. Это означает, что запрашиваемого адреса не~существует в~словаре объектов приемопередатчика или неверна комбинация начального адреса и~количества запрашиваемых параметров. Например, для устройства, имеющего 100 регистров, запрос с~начальным адресом 96 и числом регистров~4 будет корректным, а~запрос с~начальным адресом~96 и длиной~5 вызовет генерацию ошибки~02. \\ \hline
	03 (03h)	& ILLEGAL DATA VALUE					& Значение, содержащееся в~поле данных запроса, недопустимо для ведомого устройства. Это означает 0-вое или недопустимо большое количество читаемых или записываемых битовых флагов или регистров, например, 985~флагов для функции~01 в~режиме RTU.	\\ \hline
	04 (04h)	& SLAVE DEVICE FAILURE					& При попытке выполнить запрос в~ведомом устройстве произошла неисправимая ошибка. \\ \hline
%	16 (10h)	& TEMPORARILY INACCESSIBLE PARAMETER	& Введен в~данной реализации MODBUS. Редактируемый параметр недоступен в~данный момент, т.е. временно. Это означает, что редактирование (запись или модификация) запрашиваемого параметра невозможна, т.к. аппарате не~находится в~режиме <<Выведен>>. 	\\ \hline
%   	17 (11h)   	& UNCHANGEABLE PARAMETER   				& Введен в~данной реализации MODBUS. Редактируемый параметр недоступен для записи или модификации, не~редактируемый параметр. Это означает что, по~крайней мере, один из~группы параметров, запрос на~редактирование которых получен, является не~редактируемым.	\\ \hline
\end{tabularx}


	%%% ----------
\section{Управление по протоколу MODBUS} \label{sec:control}

Обработка запросов, поступающих по~сети MODBUS, выполняется в~фоновом цикле работы интерфейсного микроконтроллера приемопередатчика. Поэтому привести точное время обработки запроса невозможно. Кроме того, время обработки запроса (с~момента фиксирования конца приема кадра запроса до~момента начала передачи кадра ответа) зависит от~функции MODBUS (запись, чтение и~т.д.) и количества запрашиваемых регистров. Установлено, что среднее время обработки запроса чтения (функция 03h) составляет 200 миллисекунд. При этом время обработки запроса может доходить до~500 миллисекунд.

Следует помнить, что приведенные выше значения времени не~учитывают время передачи кадров по~сети MODBUS, которое зависит от~скорости передачи и длины передаваемого кадра.

\subsection{Дата и время}

Обращаясь к~соответствующим регистрам можно считать текущее время и дату аппарата, а~также установить новое значение. При~этом запись всех регистров необходимо \MakeUppercase{\textit{производить одной командой}}, т.е. одновременно передавать дату и время.

Запись доступна в~любое время, не~зависимо от~текущего режима работы аппаратуры.

\subsection{Пароль}

Пароль требуется для записи некоторых регистров приемопередатчика, в~том числе \textit{<<Новый пароль>>}. При этом команда на~запись регистра должна идти сразу после записи в регистр \textit{<<Пароль>>}.

\subsection{Текущее состояние аппарата}

Обращаясь к значениям соответствующих регистров, можно считать текущее состояние, режим работы, неисправности и предупреждения приемопередатчика. При~этом, надо учесть регистры неисправностей или предупреждений являются наборами флагов.

Записью нового значения \textit{<<Введен>>} или \textit{<<Выведен>>} хотя~бы в~один регистр \textit{Режим работы} можно изменить текущий режим работы приемопередатчика (необходим ввод пароля).

Дополнительно наличие неисправностей и предупреждений можно проверить по~флагам \textit{Неисправность} и \textit{Предупреждение} соответственно. Так~же имеется возможность проверить флаг для любой неисправности или предупреждения.

Значения кодов параметров приведены в приложении \ref{app:state}.

\subsection{Журналы}

Для чтения журналов команд и событий можно использовать следующий алгоритм:
\begin{enumerate}
	\item[1.] Прочитать количество записей сделанных с~момента включения аппарата.
\item[2.] Если счетчик не~изменился, значит небыло сделано ни~одной записи и можно завершить чтение.
\item[3.] Записать номер записи журнала для считывания в~параметр \textit{<<Номер текущей записи журнала>>}.
\item[4.] Читать параметр \textit{<<Номер текущей записи журнала>>} до~тех пор, пока значение не~совпадет с~записанным.
\item[5.] Считать данные записи журнала.
\item[6.] Перейти к~считыванию следующей записи, начиная с~пункта~3.
\end{enumerate}

Значения кодов параметров журналов приведены в~приложении \ref{app:jrn}.

\subsection{Текущие измерения}

Обращаясь к~соответствующим регистрам можно считать текущие значения измеряемых величин.

% Пункт описания команд есть в К400 и РЗСК
\ifCommand
\subsection{Индикация команд}

Каждый бит в~этих регистрах соответствует состоянию определенного светодиода на~панели индикации блока БСК. Младший бит это команда №1 (№17), старший бит - №16 (№32). Значение бита 0 - означает что светодиод погашен, т.е. нет~индикации прохождения данной команды, 1 - светодиод горит, зафиксировано прохождение команды.

Дополнительно имеются флаги \textit{Индикация команд передатчика} и \textit{Индикация команд приемника}, 1 в~которых означает наличие хотя~бы одной зафиксированной принятой (переданной) команды. И флаги индикации для каждой из~команд в~отдельности.

Сброс индикации осуществляется записью нуля в~любой из~этих регистров. Либо установкой~0 в~один из~флагов \textit{Индикация команды передатчика} или \textit{Индикация команды приемника}.
\fi

\subsection{Версии прошивок микросхем}

Обращаясь к~значениям соответствующих регистров, можно считать версии прошивок микросхем приемопередатчика. Старший байт значения содержит номер версии прошивки, а~младший ревизию прошивки. Например, если регистр с~адресом \textit{146 <<ПИ MCU>>} содержит значение 0х0112, то~это следует понимать как~<<01.12>>.
	%%% ----------
\section{Словарь объектов} \label{sec:map}

Словарь объектов - это список параметров (битовых флагов, регистров) приемопередатчика , параметров протокола, к~которым можно обращаться, используя коммуникационные протоколы.

Словарь объектов приемопередатчика \deviceCurrent{} представлен в~приложении \ref{app:map}.

Объекты представленные в~таблицах имеют следующие параметры:
\begin{list}{--}{}
\item адрес "--- уникальный 16-битовый цифровой идентификатор параметра для протокола MODBUS;  
\item название и~описание параметра;
\item макс. "--- максимальное значение параметра;
\item мин. "--- минимальное значение параметра;
\item масш. "--- значение единицы младшего разряда параметра;
\item доступ "--- определяет возможность чтения и записи параметра.
\end{list}

В~соответствии с протоколом MODBUS запрашиваемые/редактируемые регистры имеют размер 2~байта. Тип реальных переменных приемопередатчика может быть целочисленным или набором флагов. В~таких случаях: 
\begin{list}{--}{}
\item целочисленная переменная представлена как 2-х~байтовый регистр в~Словаре объектов. При чтении и~записи регистра его старший байт содержит значение старшего байта переменной, младший байт "--- значение младшего байта переменной. Если значение переменной допускает отрицательные величины, то она записывается в дополнительном коде (\textbf{signed}), иначе в прямом (\textbf{unsigned});
\item набор флагов представлен как 2-х~байтовый регистр в~Словаре объектов. При чтении и~записи регистра его старший байт содержит значение старшего байта переменной, младший байт "--- значение младшего байта переменной. Каждый бит регистра отвечает за~отдельное событие. Диапазон значений для них представлен в~шестнадцатеричном виде, например 0xFFFF.
\end{list}

Например, параметр 123 \textit{<<Выходное напряжение>>} представляет собой двухбайтное целое число, младший разряд которого равен 0,1~В. Принятое значение 274 это 27.4~В. А~параметр 10 \textit{<<Код неисправности Общий>>} содержит в~себе флаги текущих общих неисправностей приемопередатчика. Принятое значение 0х0021 означает, что в~текущий момент времени присутствует две неисправности c~кодами 0х0001 и~0х0002.

Необходимо помнить, что в~соответствии с протоколом MODBUS вначале передается старший байт регистра, затем младший. 

	
	\ESKDappendix{Обязательное}{Таблица Словаря объектов приемопередатчика} \label{app:map}
	% при использовании \setcounter{..} происходит сдвиг правой границы таблицы
% для того чтобы избежать этого, "\\ \hline" надо писать после него без пробелов
\newcounter{adr}
\newcommand{\cntadr}{%
\arabic{adr}\stepcounter{adr}%
}



\begin{tabularx}{\linewidth}{|M{1.2cm}|X|M{1.7cm}|M{1.7cm}|M{1.7cm}|M{1.7cm}|}	
	\caption{Таблица регистров Словаря объектов приемопередатчика} \label{tab:map_register} \\
	
	\hline
    Адрес	& \calign{Название и~описание параметра}	
    										& Масш.		& Мин.		& Макс.		& Доступ	\\ \hline
    \endfirsthead
    
    \multicolumn{6}{r}{продолжение следует\ldots} \\
    \endfoot
	\endlastfoot
	
	\multicolumn{6}{l}{Продолжение таблицы \ref{tab:map_register}} \\ \hline 	
	Адрес	&  \calign{Название и~описание параметра} 	
											& Масш.		& Мин.		& Макс.		& Доступ	\\ \hline
	\endhead
	
	\multicolumn{6}{|c|}{Дата и~время}									\setcounter{adr}{01}\\ \hline																	
	\cntadr	& Год 							& 1~год 	& 0 		& 99 		& чт./зап.	\\ \hline
	\cntadr & Месяц 						& 1~мес 	& 1 		& 12 		& чт./зап.	\\ \hline
    \cntadr	& День 							& 1~день	& 1 		& 31 		& чт./зап.	\\ \hline
    \cntadr	& Часы 							& 1~час 	& 0 		& 23 		& чт./зап.	\\ \hline
    \cntadr	& Минуты 						& 1~мин 	& 0 		& 59 		& чт./зап.	\\ \hline
    \cntadr	& Секунды					 	& 1~сек 	& 0 		& 59 		& чт./зап.	\\ \hline
    \multicolumn{6}{|c|}{Пароль}										\setcounter{adr}{07}\\ \hline
    \cntadr	& Пароль						&			& 0			& 9999		& зап.		\\ \hline 
    \cntadr	& Новый пароль					&			& 0			& 9999		& зап.		\\ \hline
    \multicolumn{6}{|c|}{Текущее состояние приемопередатчика}			\setcounter{adr}{10}\\ \hline
    10		& Код неисправности Общий 		& ---		& 0x0000 	& 0xFFFF 	& чт.		\\ \hline
    11		& Код предупреждения Общий 		& --- 		& 0x0000 	& 0xFFFF 	& чт.		\\ \hline
\ifCommand
    12 		& Код неисправности Приемника 	& --- 		& 0x0000 	& 0xFFFF 	& чт.		\\ \hline
    13 		& Код предупреждения Приемника 	& --- 		& 0x0000 	& 0xFFFF 	& чт.		\\ \hline
    14 		& Режим работы Приемника 		& 1 		& 0 		& 5 		& чт./зап.	\\ \hline
    15 		& Состояние Приемника 			& 1 		& 0 		& 12 		& чт.		\\ \hline
    16 		& Доп.~байт Приемника 			& 1 		& 0 		& 32 		& чт.		\\ \hline
    17 		& Код неисправности Передатчика & --- 		& 0x0000 	& 0xFFFF 	& чт.		\\ \hline
    18 		& Код предупрежд. Передатчика 	& --- 		& 0x0000 	& 0xFFFF 	& чт.		\\ \hline
    19 		& Режим работы Передатчика 		& 1 		& 0 		& 5 		& чт./зап.	\\ \hline
    20 		& Состояние Передатчика 		& 1 		& 0 		& 12 		& чт.		\\ \hline
	21 		& Доп.~байт Передатчика 		& 1 		& 0 		& 32 		& чт.		\\ \hline
\fi
\ifDefense
	22 		& Код неисправности Защиты		& --- 		& 0x0000 	& 0xFFFF 	& чт.		\\ \hline
	23 		& Код предупрежд. Защиты 		& --- 		& 0x0000 	& 0xFFFF 	& чт.		\\ \hline
    24 		& Режим работы Защиты 			& 1 		& 0 		& 5 		& чт./зап.	\\ \hline
    25 		& Состояние Защиты 				& 1 		& 0 		& 12 		& чт.		\\ \hline
\fi
    \multicolumn{6}{|c|}{Журнал событий}								\setcounter{adr}{27}\\ \hline
    \cntadr	& Количество записей сделанных с~момента включения аппарата 
    										& 1 		& 0 		& 65535 	& чт.		\\ \hline
    \cntadr	& Количество записей в журнале 	& 1 		& 0 		& 512 		& чт.		\\ \hline
    \cntadr	& Номер текущей записи журнала 	& 1 		& 0 		& 512 		& чт./зап.	\\ \hline
    \cntadr	& Резерв 						& --- 		& --- 		& --- 		& ---		\\ \hline
    \cntadr	& Резерв 						& --- 		& --- 		& --- 		& ---		\\ \hline
    \cntadr	& Имя устройства 				& 1 		& 1 		& 5 		& чт.		\\ \hline
    \cntadr	& Тип события 					& 1 		& 1 		& 32 		& чт.		\\ \hline
    \cntadr	& Режим работы 					& 1 		& 0 		& 5 		& чт.		\\ \hline
    \cntadr	& Резерв 						& --- 		& --- 		& --- 		& ---		\\ \hline
    \cntadr	& Резерв						& --- 		& --- 		& --- 		& ---		\\ \hline
    \cntadr	& Резерв 						& --- 		& --- 		& --- 		& ---		\\ \hline
    \cntadr	& Резерв 						& --- 		& --- 		& --- 		& ---		\\ \hline
   	\cntadr	& Миллисекунды 					& 1~мс 		& 0 		& 999 		& чт.		\\ \hline
   	\cntadr	& Секунды 						& 1~сек 	& 0 		& 59 		& чт.		\\ \hline
   	\cntadr	& Минуты 						& 1~мин 	& 0 		& 59 		& чт.		\\ \hline
   	\cntadr	& Часы 							& 1~час 	& 0 		& 23 		& чт.   	\\ \hline
 	\cntadr	& День недели 					& 1 		& 1 		& 7 		& чт.		\\ \hline
 	\cntadr	& День 							& 1~день 	& 1 		& 31 		& чт.		\\ \hline
 	\cntadr	& Месяц 						& 1~мес 	& 1 		& 12 		& чт.		\\ \hline
 	\cntadr	& Год 							& 1~год 	& 0 		& 99 		& чт.		\\ \hline 	
\ifCommand
 	\multicolumn{6}{|c|}{Журнал Приемника}								\setcounter{adr}{50}\\ \hline
 	\cntadr	& Количество записей сделанных с~момента включения аппарата 
 											& 1 		& 0 		& 65535 	& чт.		\\ \hline
 	\cntadr	& Количество записей в~журнале 	& 1 		& 0 		& 512 		& чт.		\\ \hline
 	\cntadr	& Номер текущей записи журнала 	& 1 		& 0 		& 512 		& чт./зап.	\\ \hline
 	\cntadr	& Резерв 						& --- 		& --- 		& --- 		& ---		\\ \hline
 	\cntadr	& Резерв 						& --- 		& --- 		& --- 		& ---		\\ \hline
 	\cntadr	& Имя устройства 				& 1 		& 1 		& 5 		& чт.		\\ \hline
 	\cntadr	& Номер команды 				& 1 		& 0 		& 32 		& чт.		\\ \hline
 	\cntadr	& Событие 						& 1 		& 0 		& 1 		& чт.		\\ \hline
 	\cntadr	& Резерв 						& --- 		& --- 		& --- 		& ---		\\ \hline
 	\cntadr	& Резерв						& --- 		& --- 		& --- 		& ---		\\ \hline
 	\cntadr	& Резерв 						& --- 		& --- 		& --- 		& ---		\\ \hline
 	\cntadr	& Резерв 						& --- 		& --- 		& --- 		& ---		\\ \hline
 	\cntadr	& Миллисекунды 					& 1~мс 		& 0 		& 999 		& чт.		\\ \hline
 	\cntadr	& Секунды 						& 1~сек 	& 0 		& 59 		& чт.		\\ \hline
 	\cntadr	& Минуты 						& 1~мин 	& 0 		& 59 		& чт.		\\ \hline
 	\cntadr	& Часы 							& 1~час 	& 0 		& 23 		& чт.   	\\ \hline
 	\cntadr	& День недели 					& 1 		& 1 		& 7 		& чт.		\\ \hline
 	\cntadr	& День 							& 1~день 	& 1 		& 31 		& чт.		\\ \hline
 	\cntadr	& Месяц 						& 1~мес 	& 1 		& 12 		& чт.		\\ \hline
 	\cntadr	& Год 							& 1~год 	& 0 		& 99 		& чт.		\\ \hline
\fi
\ifCommand 	
	\multicolumn{6}{|c|}{Журнал Передатчика}							\setcounter{adr}{74}\\ \hline
	\cntadr	& Количество записей сделанных с~момента включения аппарата 
 											& 1 		& 0 		& 65535 	& чт.		\\ \hline
 	\cntadr	& Количество записей в~журнале 	& 1 		& 0 		& 512 		& чт.		\\ \hline
 	\cntadr	& Номер текущей записи журнала 	& 1 		& 0 		& 512 		& чт./зап.	\\ \hline
 	\cntadr	& Резерв 						& --- 		& --- 		& --- 		& ---		\\ \hline
 	\cntadr & Резерв 						& --- 		& --- 		& --- 		& ---		\\ \hline
 	\cntadr & Имя устройства 				& 1 		& 1 		& 5 		& чт.		\\ \hline
 	\cntadr	& Номер команды 				& 1 		& 0 		& 32 		& чт.		\\ \hline
 	\cntadr	& Событие 						& 1 		& 0 		& 1 		& чт.		\\ \hline
 	\cntadr	& Источник команды				& 1 		& 0 		& 1 		& чт.		\\ \hline
 	\cntadr & Резерв						& --- 		& --- 		& --- 		& ---		\\ \hline
 	\cntadr	& Резерв 						& --- 		& --- 		& --- 		& ---		\\ \hline
 	\cntadr & Резерв 						& --- 		& --- 		& --- 		& ---		\\ \hline
 	\cntadr & Миллисекунды 					& 1~мс 		& 0 		& 999 		& чт.		\\ \hline
 	\cntadr	& Секунды 						& 1~сек 	& 0 		& 59 		& чт.		\\ \hline
 	\cntadr	& Минуты 						& 1~мин 	& 0 		& 59 		& чт.		\\ \hline
 	\cntadr	& Часы 							& 1~час 	& 0 		& 23 		& чт.   	\\ \hline
 	\cntadr	& День недели 					& 1 		& 1 		& 7 		& чт.		\\ \hline
 	\cntadr	& День 							& 1~день 	& 1 		& 31 		& чт.		\\ \hline
 	\cntadr	& Месяц 						& 1~мес 	& 1 		& 12 		& чт.		\\ \hline
 	\cntadr	& Год 							& 1~год 	& 0 		& 99 		& чт.		\\ \hline 		
\fi		
\ifDefense
	\multicolumn{6}{|c|}{Журнал Защиты}									\setcounter{adr}{98}\\ \hline
	\cntadr	& Количество записей сделанных с~момента включения аппарата 
 											& 1 		& 0 		& 65535 	& чт.		\\ \hline
 	\cntadr	& Количество записей в~журнале 	& 1 		& 0 		& 512 		& чт.		\\ \hline
 	\cntadr	& Номер текущей записи журнала 	& 1 		& 0 		& 512 		& чт./зап.	\\ \hline
 	\cntadr	& Резерв 						& --- 		& --- 		& --- 		& ---		\\ \hline
 	\cntadr & Резерв 						& --- 		& --- 		& --- 		& ---		\\ \hline
 	\cntadr & Имя устройства 				& 1 		& 1 		& 5 		& чт.		\\ \hline
 	\cntadr	& Состояние сигналов			& 1 		& 0x0000 	& 0x7FFF	& чт.		\\ \hline
 	\cntadr	& Состояние						& 1 		& 0 		& 12 		& чт.		\\ \hline
 	\cntadr	& Резерв						& --- 		& --- 		& --- 		& ---		\\ \hline
 	\cntadr & Резерв						& --- 		& --- 		& --- 		& ---		\\ \hline
 	\cntadr	& Резерв 						& --- 		& --- 		& --- 		& ---		\\ \hline
 	\cntadr & Резерв 						& --- 		& --- 		& --- 		& ---		\\ \hline
 	\cntadr & Миллисекунды 					& 1~мс 		& 0 		& 999 		& чт.		\\ \hline
 	\cntadr	& Секунды 						& 1~сек 	& 0 		& 59 		& чт.		\\ \hline
 	\cntadr	& Минуты 						& 1~мин 	& 0 		& 59 		& чт.		\\ \hline
 	\cntadr	& Часы 							& 1~час 	& 0 		& 23 		& чт.   	\\ \hline
 	\cntadr	& День недели 					& 1 		& 1 		& 7 		& чт.		\\ \hline
 	\cntadr	& День 							& 1~день 	& 1 		& 31 		& чт.		\\ \hline
 	\cntadr	& Месяц 						& 1~мес 	& 1 		& 12 		& чт.		\\ \hline
 	\cntadr	& Год 							& 1~год 	& 0 		& 99 		& чт.		\\ \hline 		
\fi				
	\multicolumn{6}{|c|}{Текущие измерения} 							\setcounter{adr}{123}\\ \hline
 	123 	& Выходное напряжение 			& 0,1~В 	& 0 		& 999 		& чт.		\\ \hline
 	124 	& Выходной ток 					& 1~мА 		& 0 		& 999 		& чт.		\\ \hline
 	125 	& Запас по~затуханию для первого приемника сигналов команд 
 											& 1~дБ 		& -99 		& 99 		& чт.		\\ \hline
 	126 	& Запас по~затуханию для второго приемника сигналов команд 
 											& 1~дБ 		& -99 		& 99 		& чт.		\\ \hline
 	127 	& Уровень принимаемого сигнала в~рабочей полосе частот первого приемника относительно чувствительности 
 											& 1~дБ 		& -99 		& 99 		& чт.		\\ \hline
 	128 	& Уровень принимаемого сигнала в~рабочей полосе частот второго приемника относительно чувствительности 
 											& 1~дБ 		& -99 		& 99 		& чт.		\\ \hline
\ifDefense
 	129 	& Запас по~затуханию для первого приемника сигналов ВЧ защит 
 											& 1~дБ 		& -99 		& 99 		& чт.		\\ \hline
 	130 	& Запас по~затуханию для второго приемника сигналов ВЧ защит 
 											& 1~дБ 		& -99 		& 99 		& чт.		\\ \hline
 	131 	& Длительность импульсов ВЧ блокировки на~выходе приемника 
 											& 1~град 	& 0 		& 360 		& чт.		\\ \hline
\fi
\ifCommand
 	\multicolumn{6}{|c|}{Индикация команд}								\setcounter{adr}{140}\\ \hline
 	140		& Индикация команд приемника №1-16		
 											& --- 		& 0x0000	& 0xFFFF	& чт./зап.	\\ \hline
\ifx \deviceCurrent \deviceK % только в К400 может быть больше 8 команд
 	141		& Индикация команд приемника №17-32		
 											& ---		& 0x0000	& 0xFFFF	& чт./зап.  \\ \hline
\fi % \if \deviceCurrent \deviceK
 	142		& Индикация команд передатчика №1-16	
 											& ---		& 0x0000	& 0xFFFF	& чт./зап.  \\ \hline
\ifx \deviceCurrent \deviceK % только в К400 может быть больше 8 команд
 	143		& Индикация команд передатчика №17-32	
 											& ---		& 0x0000	& 0xFFFF	& чт./зап.	\\ \hline
\fi % \if \deviceCurrent \deviceK
\fi					
 	\multicolumn{6}{|c|}{Версии прошивок микросхем}						\setcounter{adr}{156}\\ \hline
 	156 	& БСП MCU 						& --- 		& 0x0000 	& 0xFFFF 	& чт.		\\ \hline
 	157 	& БСП DSP 						& --- 		& 0x0000 	& 0xFFFF 	& чт.		\\ \hline
 	158 	& ПИ MCU 						& --- 		& 0x0000 	& 0xFFFF 	& чт.		\\ \hline
\ifCommand
 	159 	& БСК ПРД1 						& --- 		& 0x0000 	& 0xFFFF 	& чт.		\\ \hline	
 	160 	& БСК ПРМ1 						& --- 		& 0x0000 	& 0xFFFF 	& чт.		\\ \hline
 	161 	& БСК ПРД2 						& --- 		& 0x0000 	& 0xFFFF 	& чт.		\\ \hline
 	162 	& БСК ПРМ2 						& --- 		& 0x0000 	& 0xFFFF 	& чт.		\\ \hline
\fi
\ifDefense
	163		& БСЗ 							& ---		& 0x0000	& 0xFFFF	& чт.		\\ \hline
\fi
\end{tabularx}

\begin{tabularx}{\linewidth}{|M{1.2cm}|X|M{1.7cm}|}	
	\caption{Таблица битовых флагов Словаря объектов приемопередатчика} \label{tab:map_flags} \\
	
	\hline
	Адрес	& \calign{Название и~описание параметра}	& Доступ	\\ \hline
	\endfirsthead
	
	\multicolumn{3}{r}{продолжение следует\ldots} \\
	\endfoot
	\endlastfoot
	
	\multicolumn{3}{l}{Продолжение таблицы \ref{tab:map_flags}} 	\\ \hline 
	Адрес	& \calign{Название и~описание параметра}	& Доступ	\\ \hline
	\endhead
	
	\multicolumn{3}{|c|}{Флаги текущего состояния} 		\setcounter{adr}{201}\\ \hline
	201 	& Неисправность 							&  чт. 		\\ \hline
	202 	& Предупреждение 							&  чт.		\\ \hline
\ifCommand
	203		& Индикация команд передатчика				&  чт./зап.	\\ \hline
	204 	& Индикация команд приемника 				&  чт./зап.	\\ \hline 
\fi
			&											&			\\ \hline
	\multicolumn{3}{|c|}{Флаги неисправностей общих}	\setcounter{adr}{301}\\ \hline
	\cntadr	& \deviceGlobalErrorI						& чт.		\\ \hline
	\cntadr	& \deviceGlobalErrorII						& чт.		\\ \hline
	\cntadr	& \deviceGlobalErrorIII						& чт.		\\ \hline
	\cntadr	& \deviceGlobalErrorIV						& чт.		\\ \hline
	\cntadr	& \deviceGlobalErrorV						& чт.		\\ \hline
	\cntadr	& \deviceGlobalErrorVI						& чт.		\\ \hline
	\cntadr	& \deviceGlobalErrorVII						& чт.		\\ \hline
	\cntadr	& \deviceGlobalErrorVIII					& чт.		\\ \hline
	\cntadr	& \deviceGlobalErrorIX						& чт.		\\ \hline
	\cntadr	& \deviceGlobalErrorX						& чт.		\\ \hline
	\cntadr	& \deviceGlobalErrorXI						& чт.		\\ \hline
	\cntadr	& \deviceGlobalErrorXII						& чт.		\\ \hline
	\cntadr	& \deviceGlobalErrorXIII					& чт.		\\ \hline
	\cntadr	& \deviceGlobalErrorXIV 					& чт.		\\ \hline
	\cntadr	& \deviceGlobalErrorXV 						& чт.		\\ \hline
	\cntadr	& \deviceGlobalErrorXVI 					& чт.		\\ \hline
	\multicolumn{3}{|c|}{Флаги предупреждений общих} \setcounter{adr}{317}\\ \hline
	\cntadr	& \deviceGlobalWarnI						& чт.		\\ \hline
	\cntadr	& \deviceGlobalWarnII						& чт.		\\ \hline
	\cntadr	& \deviceGlobalWarnIII						& чт.		\\ \hline
	\cntadr	& \deviceGlobalWarnIV						& чт.		\\ \hline
	\cntadr	& \deviceGlobalWarnV						& чт.		\\ \hline
	\cntadr	& \deviceGlobalWarnVI						& чт.		\\ \hline
	\cntadr	& \deviceGlobalWarnVII						& чт.		\\ \hline
	\cntadr	& \deviceGlobalWarnVIII						& чт.		\\ \hline
	\cntadr	& \deviceGlobalWarnIX						& чт.		\\ \hline
	\cntadr	& \deviceGlobalWarnX						& чт.		\\ \hline
	\cntadr	& \deviceGlobalWarnXI						& чт.		\\ \hline
	\cntadr	& \deviceGlobalWarnXII						& чт.		\\ \hline
	\cntadr	& \deviceGlobalWarnXIII						& чт.		\\ \hline
	\cntadr	& \deviceGlobalWarnXIV 						& чт.		\\ \hline
	\cntadr	& \deviceGlobalWarnXV 						& чт.		\\ \hline
	\cntadr	& \deviceGlobalWarnXVI 						& чт.		\\ \hline	
\ifCommand
			&											&			\\ \hline	
	400		& Передатчик подключен						& чт.		\\ \hline
	\multicolumn{3}{|c|}{Флаги неисправностей Передатчика}\setcounter{adr}{401}\\ \hline	
	\cntadr	& \devicePrdErrorI							& чт.		\\ \hline
	\cntadr	& \devicePrdErrorII							& чт.		\\ \hline
	\cntadr	& \devicePrdErrorIII						& чт.		\\ \hline
	\cntadr	& \devicePrdErrorIV							& чт.		\\ \hline
	\cntadr	& \devicePrdErrorV							& чт.		\\ \hline
	\cntadr	& \devicePrdErrorVI							& чт.		\\ \hline
	\cntadr	& \devicePrdErrorVII						& чт.		\\ \hline
	\cntadr	& \devicePrdErrorVIII						& чт.		\\ \hline
	\cntadr	& \devicePrdErrorIX							& чт.		\\ \hline
	\cntadr	& \devicePrdErrorX							& чт.		\\ \hline
	\cntadr	& \devicePrdErrorXI							& чт.		\\ \hline
	\cntadr	& \devicePrdErrorXII						& чт.		\\ \hline
	\cntadr	& \devicePrdErrorXIII						& чт.		\\ \hline
	\cntadr	& \devicePrdErrorXIV 						& чт.		\\ \hline
	\cntadr	& \devicePrdErrorXV 						& чт.		\\ \hline
	\cntadr	& \devicePrdErrorXVI 						& чт.		\\ \hline	
	\multicolumn{3}{|c|}{Флаги предупреждений Передатчика}\setcounter{adr}{417}\\ \hline
	\cntadr	& \devicePrdWarnI							& чт.		\\ \hline
	\cntadr	& \devicePrdWarnII							& чт.		\\ \hline
	\cntadr	& \devicePrdWarnIII							& чт.		\\ \hline
	\cntadr	& \devicePrdWarnIV							& чт.		\\ \hline
	\cntadr	& \devicePrdWarnV							& чт.		\\ \hline
	\cntadr	& \devicePrdWarnVI							& чт.		\\ \hline
	\cntadr	& \devicePrdWarnVII							& чт.		\\ \hline
	\cntadr	& \devicePrdWarnVIII						& чт.		\\ \hline
	\cntadr	& \devicePrdWarnIX							& чт.		\\ \hline
	\cntadr	& \devicePrdWarnX							& чт.		\\ \hline
	\cntadr	& \devicePrdWarnXI							& чт.		\\ \hline
	\cntadr	& \devicePrdWarnXII							& чт.		\\ \hline
	\cntadr	& \devicePrdWarnXIII						& чт.		\\ \hline
	\cntadr	& \devicePrdWarnXIV 						& чт.		\\ \hline
	\cntadr	& \devicePrdWarnXV 							& чт.		\\ \hline
	\cntadr	& \devicePrdWarnXVI 						& чт.		\\ \hline
	\multicolumn{3}{|c|}{Флаги индикации команд Передатчика}\setcounter{adr}{450}\\ \hline
	\cntadr	& Индикация команды передатчика №1  		& чт.		\\ \hline
	\cntadr	& Индикация команды передатчика №2  		& чт.		\\ \hline
	\cntadr	& Индикация команды передатчика №3  		& чт.		\\ \hline
	\cntadr	& Индикация команды передатчика №4  		& чт.		\\ \hline
	\cntadr	& Индикация команды передатчика №5  		& чт.		\\ \hline
	\cntadr	& Индикация команды передатчика №6  		& чт.		\\ \hline
	\cntadr	& Индикация команды передатчика №7  		& чт.		\\ \hline
	\cntadr	& Индикация команды передатчика №8  		& чт.		\\ \hline
\ifx \deviceCurrent \deviceK % только в К400 может быть больше 8 команд
	\cntadr	& Индикация команды передатчика №9  		& чт.		\\ \hline
	\cntadr	& Индикация команды передатчика №10 		& чт.		\\ \hline
	\cntadr	& Индикация команды передатчика №11 		& чт.		\\ \hline	
	\cntadr	& Индикация команды передатчика №12 		& чт.		\\ \hline
	\cntadr	& Индикация команды передатчика №13 		& чт.		\\ \hline
	\cntadr	& Индикация команды передатчика №14 		& чт.		\\ \hline
	\cntadr	& Индикация команды передатчика №15 		& чт.		\\ \hline
	\cntadr	& Индикация команды передатчика №16 		& чт.		\\ \hline
	\cntadr	& Индикация команды передатчика №17 		& чт.		\\ \hline
	\cntadr	& Индикация команды передатчика №18 		& чт.		\\ \hline
	\cntadr	& Индикация команды передатчика №19 		& чт.		\\ \hline
	\cntadr	& Индикация команды передатчика №20 		& чт.		\\ \hline
	\cntadr	& Индикация команды передатчика №21 		& чт.		\\ \hline
	\cntadr	& Индикация команды передатчика №22 		& чт.		\\ \hline
	\cntadr	& Индикация команды передатчика №23 		& чт.		\\ \hline
	\cntadr	& Индикация команды передатчика №24 		& чт.		\\ \hline
	\cntadr	& Индикация команды передатчика №25 		& чт.		\\ \hline
	\cntadr	& Индикация команды передатчика №26 		& чт.		\\ \hline
	\cntadr	& Индикация команды передатчика №27 		& чт.		\\ \hline
	\cntadr	& Индикация команды передатчика №28 		& чт.		\\ \hline
	\cntadr	& Индикация команды передатчика №29 		& чт.		\\ \hline
	\cntadr	& Индикация команды передатчика №30 		& чт.		\\ \hline
	\cntadr	& Индикация команды передатчика №31 		& чт.		\\ \hline
	\cntadr	& Индикация команды передатчика №32 		& чт.		\\ \hline
\fi % \ifx \deviceCurrent \deviceK 	
\fi % \ifCommand
\ifCommand
			&											&			\\ \hline
	500		& Приемник подключен						& чт.		\\ \hline
	\multicolumn{3}{|c|}{Флаги неисправностей Приемника}\setcounter{adr}{501}\\ \hline
	\cntadr	& \devicePrmErrorI							& чт.		\\ \hline
	\cntadr	& \devicePrmErrorII							& чт.		\\ \hline
	\cntadr	& \devicePrmErrorIII						& чт.		\\ \hline
	\cntadr	& \devicePrmErrorIV							& чт.		\\ \hline
	\cntadr	& \devicePrmErrorV							& чт.		\\ \hline
	\cntadr	& \devicePrmErrorVI							& чт.		\\ \hline
	\cntadr	& \devicePrmErrorVII						& чт.		\\ \hline
	\cntadr	& \devicePrmErrorVIII						& чт.		\\ \hline
	\cntadr	& \devicePrmErrorIX							& чт.		\\ \hline
	\cntadr	& \devicePrmErrorX							& чт.		\\ \hline
	\cntadr	& \devicePrmErrorXI							& чт.		\\ \hline
	\cntadr	& \devicePrmErrorXII						& чт.		\\ \hline
	\cntadr	& \devicePrmErrorXIII						& чт.		\\ \hline
	\cntadr	& \devicePrmErrorXIV 						& чт.		\\ \hline
	\cntadr	& \devicePrmErrorXV 						& чт.		\\ \hline
	\cntadr	& \devicePrmErrorXVI 						& чт.		\\ \hline		
	\multicolumn{3}{|c|}{Флаги предупреждений Приемника}\setcounter{adr}{517}\\ \hline	
	\cntadr	& \devicePrmWarnI							& чт.		\\ \hline
	\cntadr	& \devicePrmWarnII							& чт.		\\ \hline
	\cntadr	& \devicePrmWarnIII							& чт.		\\ \hline
	\cntadr	& \devicePrmWarnIV							& чт.		\\ \hline
	\cntadr	& \devicePrmWarnV							& чт.		\\ \hline
	\cntadr	& \devicePrmWarnVI							& чт.		\\ \hline
	\cntadr	& \devicePrmWarnVII							& чт.		\\ \hline
	\cntadr	& \devicePrmWarnVIII						& чт.		\\ \hline
	\cntadr	& \devicePrmWarnIX							& чт.		\\ \hline
	\cntadr	& \devicePrmWarnX							& чт.		\\ \hline
	\cntadr	& \devicePrmWarnXI							& чт.		\\ \hline
	\cntadr	& \devicePrmWarnXII							& чт.		\\ \hline
	\cntadr	& \devicePrmWarnXIII						& чт.		\\ \hline
	\cntadr	& \devicePrmWarnXIV 						& чт.		\\ \hline
	\cntadr	& \devicePrmWarnXV 							& чт.		\\ \hline
	\cntadr	& \devicePrmWarnXVI 						& чт.		\\ \hline
	\multicolumn{3}{|c|}{Флаги неисправностей Приемника}\setcounter{adr}{550}\\ \hline
	\cntadr	& Индикация команды приемника №1  			& чт.		\\ \hline
	\cntadr	& Индикация команды приемника №2  			& чт.		\\ \hline
	\cntadr	& Индикация команды приемника №3  			& чт.		\\ \hline
	\cntadr	& Индикация команды приемника №4  			& чт.		\\ \hline
	\cntadr	& Индикация команды приемника №5  			& чт.		\\ \hline
	\cntadr	& Индикация команды приемника №6  			& чт.		\\ \hline
	\cntadr	& Индикация команды приемника №7  			& чт.		\\ \hline
	\cntadr	& Индикация команды приемника №8  			& чт.		\\ \hline
\ifx \deviceCurrent \deviceK % в К400 может быть больше 8 команд
	\cntadr	& Индикация команды приемника №9  			& чт.		\\ \hline
	\cntadr	& Индикация команды приемника №10 			& чт.		\\ \hline
	\cntadr	& Индикация команды приемника №11 			& чт.		\\ \hline
	\cntadr	& Индикация команды приемника №12 			& чт.		\\ \hline
	\cntadr	& Индикация команды приемника №13 			& чт.		\\ \hline
	\cntadr	& Индикация команды приемника №14 			& чт.		\\ \hline
	\cntadr	& Индикация команды приемника №15 			& чт.		\\ \hline
	\cntadr	& Индикация команды приемника №16 			& чт.		\\ \hline
	\cntadr	& Индикация команды приемника №17 			& чт.		\\ \hline
	\cntadr	& Индикация команды приемника №18 			& чт.		\\ \hline
	\cntadr	& Индикация команды приемника №19 			& чт.		\\ \hline
	\cntadr	& Индикация команды приемника №20 			& чт.		\\ \hline
	\cntadr	& Индикация команды приемника №21 			& чт.		\\ \hline
	\cntadr	& Индикация команды приемника №22 			& чт.		\\ \hline
	\cntadr	& Индикация команды приемника №23 			& чт.		\\ \hline
	\cntadr	& Индикация команды приемника №24 			& чт.		\\ \hline
	\cntadr	& Индикация команды приемника №25 			& чт.		\\ \hline
	\cntadr	& Индикация команды приемника №26 			& чт.		\\ \hline
	\cntadr	& Индикация команды приемника №27 			& чт.		\\ \hline
	\cntadr	& Индикация команды приемника №28 			& чт.		\\ \hline
	\cntadr	& Индикация команды приемника №29 			& чт.		\\ \hline
	\cntadr	& Индикация команды приемника №30 			& чт.		\\ \hline
	\cntadr	& Индикация команды приемника №31 			& чт.		\\ \hline
	\cntadr	& Индикация команды приемника №32 			& чт.		\\ \hline
\fi % \ifx \deviceCurrent \deviceK 	
\fi % \ifCommand
\ifDefense
			&											&			\\ \hline
	500		& Защита подключена							& чт.		\\ \hline
	\multicolumn{3}{|c|}{Флаги неисправностей Защиты} \setcounter{adr}{601}\\ \hline
	\cntadr	& \deviceDefenseErrorI						& чт.		\\ \hline
	\cntadr	& \deviceDefenseErrorII						& чт.		\\ \hline
	\cntadr	& \deviceDefenseErrorIII					& чт.		\\ \hline
	\cntadr	& \deviceDefenseErrorIV						& чт.		\\ \hline
	\cntadr	& \deviceDefenseErrorV						& чт.		\\ \hline
	\cntadr	& \deviceDefenseErrorVI						& чт.		\\ \hline
	\cntadr	& \deviceDefenseErrorVII					& чт.		\\ \hline
	\cntadr	& \deviceDefenseErrorVIII					& чт.		\\ \hline
	\cntadr	& \deviceDefenseErrorIX						& чт.		\\ \hline
	\cntadr	& \deviceDefenseErrorX						& чт.		\\ \hline
	\cntadr	& \deviceDefenseErrorXI						& чт.		\\ \hline
	\cntadr	& \deviceDefenseErrorXII					& чт.		\\ \hline
	\cntadr	& \deviceDefenseErrorXIII					& чт.		\\ \hline
	\cntadr	& \deviceDefenseErrorXIV 					& чт.		\\ \hline
	\cntadr	& \deviceDefenseErrorXV 					& чт.		\\ \hline
	\cntadr	& \deviceDefenseErrorXVI 					& чт.		\\ \hline	
	\multicolumn{3}{|c|}{Флаги предупреждений Защиты} \setcounter{adr}{617}\\ \hline	
	\cntadr	& \deviceDefenseWarnI						& чт.		\\ \hline
	\cntadr	& \deviceDefenseWarnII						& чт.		\\ \hline
	\cntadr	& \deviceDefenseWarnIII						& чт.		\\ \hline
	\cntadr	& \deviceDefenseWarnIV						& чт.		\\ \hline
	\cntadr	& \deviceDefenseWarnV						& чт.		\\ \hline
	\cntadr	& \deviceDefenseWarnVI						& чт.		\\ \hline
	\cntadr	& \deviceDefenseWarnVII						& чт.		\\ \hline
	\cntadr	& \deviceDefenseWarnVIII					& чт.		\\ \hline
	\cntadr	& \deviceDefenseWarnIX						& чт.		\\ \hline
	\cntadr	& \deviceDefenseWarnX						& чт.		\\ \hline
	\cntadr	& \deviceDefenseWarnXI						& чт.		\\ \hline
	\cntadr	& \deviceDefenseWarnXII						& чт.		\\ \hline
	\cntadr	& \deviceDefenseWarnXIII					& чт.		\\ \hline
	\cntadr	& \deviceDefenseWarnXIV 					& чт.		\\ \hline
	\cntadr	& \deviceDefenseWarnXV 						& чт.		\\ \hline
	\cntadr	& \deviceDefenseWarnXVI 					& чт.		\\ \hline	
\fi % \ifDefense
\end{tabularx}

	
	\ESKDappendix{Обязательное}{Значения параметров состояния приемопередатчика} \label{app:state}
	\begin{tabularx}{\linewidth}{| M{1.5cm} | X | m{6cm} |}
%	\caption{Значения параметров текущего состояния приемопередатчика} \label{tab:map_state} \\
	\hline
	Код & \centering Значение \arraybackslash & \centering Комментарий \arraybackslash \\ \hline
	\endfirsthead
	\multicolumn{3}{l}{Продолжение таблицы} \\ \hline 
	Код	& \centering Значение \arraybackslash & \centering Комментарий \arraybackslash \\ \hline
	\endhead
	\multicolumn{3}{r}{продолжение следует\ldots} 
	\endfoot
	\endlastfoot
	
	\multicolumn{3}{|c|}{Режим работы} 									\\ \hline
	0 		& Выведен							& 						\\ \hline
	1 		& Готов 							& 						\\ \hline 
	2 		& Введен 							& 						\\ \hline
	3 		& Речь 								& 						\\ \hline
	4 		& Тест 								& 						\\ \hline
	5 		& Тест 								&						\\ \hline
\ifCommand
	\multicolumn{3}{|c|}{Состояние Приемника} 							\\ \hline
	0 		& Исходное							& 						\\ \hline
	1 		& Прием контрольной частоты 		& Номер в доп.байте		\\ \hline 
	2 		& Прием команды ПА  				& Номер в доп.байте		\\ \hline
	3 		& Нет КЧ							& 						\\ \hline
	4 		& Неисправность 					& 						\\ \hline
	5 		& Полная неисправность 				&						\\ \hline
	6 		& Ожидание 							&						\\ \hline
	7 		& Блокированная команда				& Номер в доп.байте		\\ \hline
	8 		& Резерв			 				&						\\ \hline
	9 		& Резерв 							&						\\ \hline
	10 		& Речь 								&						\\ \hline
	11 		& ПРД 								& Для режимов Тест		\\ \hline
	12 		& ПРМ 								& Для режимов Тест		\\ \hline	
\fi
\ifCommand
	\multicolumn{3}{|c|}{Состояние Передатчика}							\\ \hline
	0 		& Исходное							& 						\\ \hline
	1 		& Передача контрольной частоты 		& Номер в доп.байте		\\ \hline 
	2 		& Передача команды ПА  				& Номер в доп.байте		\\ \hline
	3 		& Нет КЧ							& 						\\ \hline
	4 		& Неисправность 					& 						\\ \hline
	5 		& Полная неисправность 				&						\\ \hline
	6 		& Ожидание 							&						\\ \hline
	7 		& Резерв							& 						\\ \hline
	8 		& Передача команды ЦС 				& Номер в доп.байте		\\ \hline
	9 		& Резерв 							&						\\ \hline
	10 		& Речь 								&						\\ \hline
	11 		& ПРД 								& Для режимов Тест		\\ \hline
	12 		& ПРМ 								& Для режимов Тест		\\ \hline
\fi
\ifDefense
	\multicolumn{3}{|c|}{Состояние Защиты}								\\ \hline
	0 		& Исходное							& 						\\ \hline
	1 		& Контроль 							& 						\\ \hline 
	2 		& Пуск  							& 						\\ \hline
	3 		& Останов							& 						\\ \hline
	4 		& Неисправность 					& 						\\ \hline
	5 		& Полная неисправность 				&						\\ \hline
	6 		& Ожидание 							&						\\ \hline
	7 		& Наладочный пуск					& 						\\ \hline
	8 		& Удаленный пуск					& 						\\ \hline
	9 		& Нет РЗ 							&						\\ \hline
	10 		& Речь 								&						\\ \hline
	11 		& ПРД 								& Для режимов Тест		\\ \hline
	12 		& ПРМ 								& Для режимов Тест		\\ \hline
\fi
\end{tabularx}
	
	\ESKDappendix{Обязательное}{Значения параметров журналов приемопередатчика} \label{app:jrn}
	\begin{tabularx}{\linewidth}{| M{1.5cm} | X | m{5cm} |}
%	\caption{Значения параметров журнала событий} \label{tab:map_jrn_event} \\
	\hline
	Код & \centering Значение \arraybackslash & \centering Комментарий \arraybackslash \\ \hline
	\endfirsthead
	
	\multicolumn{3}{r}{продолжение следует\ldots} 
	\endfoot
	\endlastfoot
	
	\multicolumn{3}{l}{Продолжение таблицы} 		\\ \hline 
	Код	& \calign{Значение} 					& \calign{Комментарий} 	\\ \hline
	\endhead
	\multicolumn{3}{|c|}{Имя устройства} 			\\ \hline
	1 	& Приемник 1 							& \multirow{5}{\linewidth}{Устройство в аппарате, для которого была сделана запись} 	\\ \cline{1-2}
	2 	& Приемник 2 							& 		\\ \cline{1-2}
	3 	& Передатчик 							& 		\\ \cline{1-2}
	4 	& Общее 								& 		\\ \cline{1-2}
	5 	& Приемники 1 и 2 						& 	\\ \hline
	\multicolumn{3}{|c|}{Режим работы} 				\\ \hline
	0 	& Выведен								& \multirow{6}{\linewidth}{Режим работы устройства в момент когда была сделана запись} 	\\ \cline{1-2}
	1 	& Готов 								& 		\\ \cline{1-2} 
	2 	& Введен 								& 		\\ \cline{1-2}
	3 	& Речь 									& 		\\ \cline{1-2}
	4 	& Тест 									& 		\\ \cline{1-2}
	5 	& Тест 									&	\\ \hline
	\multicolumn{3}{|c|}{Источник команды}			\\ \hline
	0	& Дискретный вход						& 		\\ \cline{1-2}
	1	& Цифровой стык							&	\\ \hline
	\multicolumn{3}{|c|}{Тип события}				\\ \hline
	1	& Включение питания или перезагрузка 	& 	\\ \hline
	2  	& Выключение питания 					& 	\\ \hline
	3  	& Изменение режима работы 				& 	\\ \hline
	4  	& Резерв 								& 	\\ \hline
	5  	& Неисправность теста Передатчика 		& 	\\ \hline
	6  	& Неисправность теста Приемника 		& 	\\ \hline
	7  	& Неисправность блока БСЗ 				& 	\\ \hline
	8  	& Неисправность блока БСК 				& 	\\ \hline
	9  	& Неисправность переключателей БСЗ 		& 	\\ \hline
	10 	& Нет сигнала манипуляции 				& 	\\ \hline
	11 	& Неисправность выходной цепи 			& 	\\ \hline
	12 	& Нет сигнала РЗ 						& 	\\ \hline
	13 	& Отсутствие сигнала Пуск 				& 	\\ \hline
	14 	& Отсутствие сигнала Останов 			& 	\\ \hline
	15 	& Неисправность чтения команд 			& 	\\ \hline
	16 	& Резерв 								& 	\\ \hline	
	17 	& Неисправность работы DSP 				& 	\\ \hline
	18 	& Восстановление работы DSP 			& 	\\ \hline
	19 	& Низкое напряжение выхода 				& 	\\ \hline
	20 	& Высокое напряжение выхода 			& 	\\ \hline
	21 	& Нет КЧ 5 секунд 						& 	\\ \hline
	22 	& Резерв 								& 	\\ \hline
	23 	& Восстановление КЧ 					& 	\\ \hline
	24 	& Резерв 								& 	\\ \hline
	25 	& Неисправность чтения/записи 2RAM 		& 	\\ \hline
	26 	& Неисправность чтения/записи ПЛИС		& 	\\ \hline
	27 	& Неисправность чтения/записи FLASH 	& 	\\ \hline
	28 	& Неисправность часов 					& 	\\ \hline
	29 	& Снижение уровня КЧ 					& 	\\ \hline
	30 	& Ошибка работы ЦС 						& 	\\ \hline
	31 	& Вход RX ЦС пуст 						& 	\\ \hline
	32 	& Работа ЦС восстановлена 				& 	\\ \hline	
\end{tabularx}
	
	\input{change}

\end{document}

