%%% ----------
\section{Управление по протоколу MODBUS} \label{sec:control}

Обработка запросов, поступающих по~сети MODBUS, выполняется в~фоновом цикле работы интерфейсного микроконтроллера приемопередатчика. Поэтому привести точное время обработки запроса невозможно. Кроме того, время обработки запроса (с~момента фиксирования конца приема кадра запроса до~момента начала передачи кадра ответа) зависит от~функции MODBUS (запись, чтение и~т.д.) и количества запрашиваемых регистров. Установлено, что среднее время обработки запроса чтения (функция 03h) составляет 200 миллисекунд. При этом время обработки запроса может доходить до~500 миллисекунд.

Следует помнить, что приведенные выше значения времени не~учитывают время передачи кадров по~сети MODBUS, которое зависит от~скорости передачи и длины передаваемого кадра.

\subsection{Дата и время}

Обращаясь к~соответствующим регистрам можно считать текущее время и дату аппарата, а~также установить новое значение. При~этом запись всех регистров необходимо \MakeUppercase{\textit{производить одной командой}}, т.е. одновременно передавать дату и время.

Запись доступна в~любое время, не~зависимо от~текущего режима работы аппаратуры.

\subsection{Пароль}

Пароль требуется для записи некоторых регистров приемопередатчика, в~том числе \textit{<<Новый пароль>>}. При этом команда на~запись регистра должна идти сразу после записи в регистр \textit{<<Пароль>>}.

\subsection{Текущее состояние аппарата}

Обращаясь к значениям соответствующих регистров, можно считать текущее состояние, режим работы, неисправности и предупреждения приемопередатчика. При~этом, надо учесть регистры неисправностей или предупреждений являются наборами флагов.

Записью нового значения \textit{<<Введен>>} или \textit{<<Выведен>>} хотя~бы в~один регистр \textit{Режим работы} можно изменить текущий режим работы приемопередатчика (необходим ввод пароля).

Дополнительно наличие неисправностей и предупреждений можно проверить по~флагам \textit{Неисправность} и \textit{Предупреждение} соответственно. Так~же имеется возможность проверить флаг для любой неисправности или предупреждения.

Значения кодов параметров приведены в приложении \ref{app:state}.

\subsection{Журналы}

Для чтения журналов команд и событий можно использовать следующий алгоритм:
\begin{enumerate}
	\item[1.] Прочитать количество записей сделанных с~момента включения аппарата.
\item[2.] Если счетчик не~изменился, значит небыло сделано ни~одной записи и можно завершить чтение.
\item[3.] Записать номер записи журнала для считывания в~параметр \textit{<<Номер текущей записи журнала>>}.
\item[4.] Читать параметр \textit{<<Номер текущей записи журнала>>} до~тех пор, пока значение не~совпадет с~записанным.
\item[5.] Считать данные записи журнала.
\item[6.] Перейти к~считыванию следующей записи, начиная с~пункта~3.
\end{enumerate}

Значения кодов параметров журналов приведены в~приложении \ref{app:jrn}.

\subsection{Текущие измерения}

Обращаясь к~соответствующим регистрам можно считать текущие значения измеряемых величин.

% Пункт описания команд есть в К400 и РЗСК
\ifCommand
\subsection{Индикация команд}

Каждый бит в~этих регистрах соответствует состоянию определенного светодиода на~панели индикации блока БСК. Младший бит это команда №1 (№17), старший бит - №16 (№32). Значение бита 0 - означает что светодиод погашен, т.е. нет~индикации прохождения данной команды, 1 - светодиод горит, зафиксировано прохождение команды.

Дополнительно имеются флаги \textit{Индикация команд передатчика} и \textit{Индикация команд приемника}, 1 в~которых означает наличие хотя~бы одной зафиксированной принятой (переданной) команды. И флаги индикации для каждой из~команд в~отдельности.

Сброс индикации осуществляется записью нуля в~любой из~этих регистров. Либо установкой~0 в~один из~флагов \textit{Индикация команды передатчика} или \textit{Индикация команды приемника}.
\fi

\subsection{Версии прошивок микросхем}

Обращаясь к~значениям соответствующих регистров, можно считать версии прошивок микросхем приемопередатчика. Старший байт значения содержит номер версии прошивки, а~младший ревизию прошивки. Например, если регистр с~адресом \textit{146 <<ПИ MCU>>} содержит значение 0х0112, то~это следует понимать как~<<01.12>>.