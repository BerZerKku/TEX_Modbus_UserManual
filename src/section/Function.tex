%%%----------
\section{Функции и коды исключения Modbus} \label{sec:func}

В~таблице \ref{tab:function} приведены поддерживаемые программным обеспечением приемопередатчика  стандартные функции MODBUS. Подробное описание этих функций можно найти в~документе «MODBUS Application Protocol Specification» и по~адресу \url{www.modbus.org}.

\begin{tabularx}{\linewidth}{|X|M{2cm}|}
	\caption{Поддерживаемые стандартные функции MODBUS}  \label{tab:function}	\\ 
    
    \hline
    \centering Описание функции \arraybackslash	& Код функции	\\ \hline
    \endfirsthead
    
    \hline
    \multicolumn{2}{r}{продолжение следует\ldots} \\
    \endfoot
	\endlastfoot
	
	\multicolumn{2}{l}{Продолжение таблицы \ref{tab:function}} \\ \hline
	\centering Описание функции \arraybackslash																		& Код функции \\
	\endhead    
	
    Чтение внутренних битовых флагов приемопередатчика.			    												& 01 (01h)  \\ \hline
    Чтение 16-битовых регистров внешних входных сигналов или внутренних 16-битовых регистров устройства. 			& 03 (03h)	\\ \hline
    Запись битового значения во~внутренний битовый флаг приемопередатчика.											& 05 (05h)	\\ \hline	
    Запись значения в~один 16-битовый регистр внешних сигналов или 16-битовый внутренний регистр устройства.		& 06 (06h)	\\ \hline
    Запись значений битовых значений во~внутренние битовые флаги приемопередатчика.									& 10 (0Ah)	\\ \hline
    Запись значений в~16-битовые регистры внешних входных сигналов или внутренние 16-битовые регистры устройства.   & 16 (10h)	\\ \hline
    Чтение идентификатора подчиненного устройства.   																& 17 (11h)	\\ \hline
\end{tabularx}

Количество регистров в~одном запросе не~должно превышать~32, а флагов "---~256.

В таблице \ref{tab:code_error} приведены коды исключения - сообщения об~ошибках, возвращаемые приемопередатчиком в~ответ на~некорректный запрос со~стороны ведущего устройства.

\begin{tabularx}{\linewidth}{| M{1.2cm} | M{4.5cm} | X |}
	\caption{Коды исключения} \label{tab:code_error}	\\ 	 
	
    \hline
    Код	& \centering Имя \arraybackslash	& \calign{Описание кода исключения} \\ \hline
    \endfirsthead
    
    \multicolumn{3}{r}{продолжение следует\ldots} \\
    \endfoot
	\endlastfoot
	
	\multicolumn{3}{l}{Продолжение таблицы \ref{tab:code_error}} 	\\ \hline
	Код 		& Имя									& \calign{Описание кода исключения} \\ \hline
	\endhead	
	01 (01h)	& ILLEGAL FUNCTION						& Код функции, принятой в~запросе, не~поддерживается ведомым устройством. Это означает, что запрашиваемая функция не~поддерживается приемопередатчиком.	\\ \hline
	02 (02h)	& ILLEGAL DATA ADDRESS					& Адрес данных, принятый в~запросе, недоступен в~ведомом устройстве. Это означает, что запрашиваемого адреса не~существует в~словаре объектов приемопередатчика или неверна комбинация начального адреса и~количества запрашиваемых параметров. Например, для устройства, имеющего 100 регистров, запрос с~начальным адресом 96 и числом регистров~4 будет корректным, а~запрос с~начальным адресом~96 и длиной~5 вызовет генерацию ошибки~02. \\ \hline
	03 (03h)	& ILLEGAL DATA VALUE					& Значение, содержащееся в~поле данных запроса, недопустимо для ведомого устройства. Это означает 0-вое или недопустимо большое количество читаемых или записываемых битовых флагов или регистров, например, 985~флагов для функции~01 в~режиме RTU.	\\ \hline
	04 (04h)	& SLAVE DEVICE FAILURE					& При попытке выполнить запрос в~ведомом устройстве произошла неисправимая ошибка. \\ \hline
%	16 (10h)	& TEMPORARILY INACCESSIBLE PARAMETER	& Введен в~данной реализации MODBUS. Редактируемый параметр недоступен в~данный момент, т.е. временно. Это означает, что редактирование (запись или модификация) запрашиваемого параметра невозможна, т.к. аппарате не~находится в~режиме <<Выведен>>. 	\\ \hline
%   	17 (11h)   	& UNCHANGEABLE PARAMETER   				& Введен в~данной реализации MODBUS. Редактируемый параметр недоступен для записи или модификации, не~редактируемый параметр. Это означает что, по~крайней мере, один из~группы параметров, запрос на~редактирование которых получен, является не~редактируемым.	\\ \hline
\end{tabularx}

