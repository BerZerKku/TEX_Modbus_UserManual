%%% ----------
\section{Установка параметров соединения} \label{sec:setup}

Параметры соединения должны быть настроены до~установки связи. 

Параметры физической линии:
\begin{list}{--}{}
	\item \textit{Baud Rate} "--- скорость передачи данных, бит/с; 
	\item \textit{Parity} "--- способ использования бита четности (<<нет>>, <<чет>>, <<нечет>>); 
	\item \textit{CL} "--- длина поля данных в посылке последовательного порта (7 или 8~бит); 
	\item \textit{SBL} --- длина поля стоп-битов в~посылке последовательного порта (1 или 2~бита);
\end{list}

Параметры протокола:
\begin{list}{--}{}
	\item \textit{Адрес в~локальной сети} - уникальный идентификационный номер устройства в~сети MODBUS.
\end{list}

Параметр физической линии \textit{CL} в~приемопередатчике равен 8~бит, и не~может быть изменен. При включенной проверке бита четности \textit{Parity} (<<чет>> или <<нечет>>), количество стоп-битов \textit{SBL} должно  быть равно~1. Если бит четности \textit{Parity} не~используется (<<нет>>), количество стоп-битов \textit{SBL} должно быть~2.  

Настройки со~стороны ведущего (master) устройства должны совпадать с~настройками приемопередатчика.

Параметр \textit{Адрес в локальной сети} приемопередатчика должен иметь уникальное, отличное от~любого другого устройства в~сети MODBUS значение в~диапазоне от~1 до~247.

Настройка параметров соединения производится с~пульта управления блока БСП в~меню <<Настройка/Интерфейс>>. Параметры соединения хранятся в~ПЗУ и не~требуют повторной настройки при~следующем включении питания.

Изменение параметров соединения с~пульта управления при~установленном соединении приводит к~потере связи.

В~таблице \ref{tab:connection} приведены возможные значения параметров и значения, установленные на~предприятии-изготовителе.

\begin{tabularx}{\linewidth}{| M{2.5cm} | X | M{4cm} | M{2cm} |}
	\caption{Параметры соединения} \label{tab:connection} \\
    
    \hline
    Название параметра	& \calign{Описание параметра}	& Возможные значения & Значение \\ \hline 
    \endfirsthead
    
    \multicolumn{4}{r}{продолжение следует\ldots} \\ 
    \endfoot 
	\endlastfoot
	
	\hline
	Название параметра		& \calign{Описание параметра}												& Возможные значения 													& Значение 	\\ \hline 
	\endhead
	Адрес в локальной сети 	& Задает уникальный идентификационный номер приемопередатчика в~сети MODBUS & от~1 до~247 															& 1			\\ \hline
	Режим MODBUS			& Позволяет выбрать формат кадра протокола MODBUS 							& только RTU															& RTU		\\ \hline
	Baud Rate				& Определяет скорость передачи данных в~сети MODBUS 						& 600, 1200, 2400, 4800, 9600, 19200 бит/с								& 19200		\\ \hline
	Parity 					& Определяет способ использования бита четности в~байтах кадра				& чет (дополнение до~четности), нечет (дополнение до~нечетности), нет 	& чет		\\ \hline
	CL						& Длина поля данных в посылке последовательного порта 						& не~настраиваются 														& 8~бит		\\ \hline
	SBL						& Длина поля стоп-битов в посылке последовательного порта 					& 1 или 2	 															& 1~бит		\\ \hline
\end{tabularx}

